\section{Deep Q-Learning}

\subsection{Principe du Deep Q-Learning}

Le $Q$-Learning requiert de stocker les valeurs de la $Q$-function de quelque manière que ce soit. Notamment, on a utilisé un tableau (la $Q$-table) dans le jeu des
bâtonnets. Néanmoins, on peut être confronté à des espaces des états de très grande taille : par exemple si les états sont des images (même de tailles raisonnables), 
alors on peut imaginer des milliards d'états possibles. Bien que généralement plus restreint, le nombre d'actions est également à l'origine de ce problème puisque la
taille de la matrice est le produit entre le nombre d'états différents et le nombre d'actions possibles. Puisque notre IA jouant à Pong doit recevoir des images du jeu,
il est impensable de stocker chaque valeur en mémoire. Le Deep $Q$-Learning résout ce problème en remplaçant la $Q$-table par un réseau neuronal. Son objectif est alors 
d'interpoler la fonction de satisfaction. Cette approximation nous permet d'économiser une importante quantité de mémoire.

On a alors un réseau neuronal qui reçoit en entrée un état et une action pour retourner la satisfaction qui y est associée. Pour pouvoir choisir une action selon
notre algorithme favorisant la plus grande satisfaction, il est nécessaire de connaître la valeur du réseau lorsqu'excité par toutes les actions. Ainsi
le nombre d'évaluations du réseau est égal au nombre d'actions possibles. Cela est d'autant plus inefficace qu'il sera nécessaire d'exécuter cette opération très
régulièrement. Pour régler ce problème, notre réseau ne reçoit plus que l'état en question en entrée. En sortie, le réseau nous donne les satisfactions associées à chaque 
état, de sorte qu'une seule évaluation du réseau suffise à choisir l'action à effectuer.

Lorsque nous réalisons l'apprentissage, nous sélectionnons une transition qui contient l'action choisie à l'instant de la transition. Puisque notre réseau renvoie
toutes les satisfactions, il est important de veiller à ce que le réseau ne réalise la rétropropagation que sur l'action de la transition. Cela peut être 
effectué en remplaçant la valeur voulue pour les autres actions par la valeur calculée, de sorte que l'erreur soit nulle, sauf pour l'action de la transition.
Pour cette dernière, on conserve l'équation \ref{eq:bellman} pour l'expression voulue de la fonction de satisfaction.

\subsection{Jeu des bâtonnets}

Pour valider notre implémentation du Deep $Q$-Learning, on le teste sur un jeu simple que l'on sait résoudre : le jeu des bâtonnets. Les règles du jeu sont les mêmes 
que précedemment. Pour la rétropropagation, nous utilisons l'optimiseur Adam particulièrement efficace, avec le taux d'apprentissage 0.001. Le facteur d'actualisation
est conservé, de même que les taux d'exploration. On stocke un maximum de 10000 transitions. Après chaque partie jouée, on apprend sur \emph{toutes} les transitions
enregistrées jusqu'à lors (par minibatch de 32 transitions). Le réseau utilisé possède 4 neurones d'entrée (chaque état est codé binairement), 10 neurones de couche cachée
puis 3 neurones de sortie pour les 3 différentes actions fournies par le jeu. Ce réseau sert à la fois à apprendre sur les transitions et à générer la cible, 
conformément à l'équation \ref{eq:bellman}. Bien que cela ne soit pas l'implémentation la plus optimale, elle suffit pour un jeu aussi simple que celui des bâtonnets.

Notre réseau apprend sur 2000 parties (contre une IA qui est également en train d'apprendre). Pour nous persuader de l'efficacité de notre algorithme, nous reconstituons
la $Q$-table à l'issue de l'apprentissage en interrogeant le réseau sur les différents états accessibles lors d'une partie. Le résultat est le suivant :

\begin{figure}[h]
\centering
\begin{tabular}{|c|c|c|c|}
\hline
Bâtonnets restants & Prendre 1 bâtonnet & Prendre 2 bâtonnets & Prendre 3 bâtonnets\\
\hline
2 & \cellcolor{green} 1.0102367 & -0.9285556 & -0.9973395\\
\hline
3 & 0.1279521 & \cellcolor{green} 1.0385635 & -1.001752\\
\hline
4 & -0.17204374 & -0.1530217 & \cellcolor{green} 1.0058613\\
\hline
5 & -0.509134 & -0.48264277 & -0.49259257\\
\hline
6 & \cellcolor{green} 0.9031452 & -0.11602497 & -0.24787879\\
\hline
7 & 0.10454643 & \cellcolor{green} 0.92308414 & -0.1744315\\
\hline
8 & 0.12821698 & 0.2924267 & \cellcolor{green} 0.9102776\\
\hline
9 & -0.06348622 & -0.0588069 & -0.05864894\\
\hline
10 & \cellcolor{green} 0.82183206 & 0.18388963 & 0.5014814\\
\hline
11 & 0.3764193 & \cellcolor{green} 0.82838273 & 0.14708841\\
\hline
12 & 0.26232004 & 0.44059992 & \cellcolor{green} 0.8352009\\
\hline
\end{tabular}
\caption{Fonction de satisfaction $Q$ approximée par le réseau neuronal}
\label{tab:Qfunction}
\end{figure}

Le tableau \ref{tab:Qfunction} nous prouve que le réseau neuronal a compris la stratégie et l'applique. En effet, pour chaque état, la satisfaction maximale est donnée
au coup prédit par la stratégie. Certaines lignes n'ont pas de maximum évident, car cela correspond à des états où la stratégie nous donne perdant, ainsi les différents
coups se valent. On remarque la satisfaction dépasse parfois en valeur absolue la récompense effectivement perçue par l'IA, cela peut paraître absurde puisque cette 
récompense est l'objectif final, mais peut s'expliquer par le fait que le réseau ne fait qu'une approximation, et ne calcule donc pas la $Q$-function exactement.
Néanmoins l'erreur reste assez faible. Enfin, on remarque que la satisfaction accordée aux bons coups est meilleure lorsque l'on est proche de la fin de la partie.
Cela provient du fait que la propagation de la récompense favorise les états proches de celui où la satisfaction est effectivement attribuée. Quoiqu'il en soit, notre IA
respecte parfaitement la stratégie, qu'elle a pu apprendre en moins de 10 minutes malgré une implémentation imparfaite.


