\section*{Introduction}
\addcontentsline{toc}{section}{Introduction}
\fancyhead[R]{\textit{Introduction}}

Le projet long Ready Player One a pour but d’étudier le fonctionnement d’algorihtmes de Machine Learning. Cette étude, orientée recherche et dévellopement, cherche à appliquer une branche du machine learning, le Q-Learning, à l’intelligence artificielle d’un jeu video, Pong. Cette IA devra pouvoir jouer au jeu le mieux possible, en se formant au fur et à mesure.

Le projet est mené par deux groupes de quatre élèves, afin de pouvoir comparer les performances des deux produits finaux. Notre groupe, le groupe « Eponge », est composé de Raphaël Bolut, Nathan Cassereau, Thomas Estiez, et Paul Le Grand Des Cloizeaux. 

Les 2 groupes sont encadrés par Mme Tomasik et M. Rimmel, qui nous donnent les instructions et les pistes à suivre pour l’aboutissement du projet, et à qui nous rendont des comptes chaque semaines sur le travail réalisé.

L’étude du projet se fait en plusieurs parties : en effet, comme la tâche à réaliser est complexe et dense, et que le projet à pour but de nous faire comprendre les mécanismes du machine learning, nous étudirons plusieurs algorithmes différents au fur et à mesure de l’année, avec lesquels nous expérimenterons : 

- dans un premier temps, nous allons étudier le fonctionnement du perceptron, un réseau de neurones basique, que nous allons entrainer à la reconnaissance de chiffres manuscrits de la base de données MNIST de Yann LeCun. Cette première étude a pour but de nous faire comprendre le fonctionnement global du machine learning, et les différents mécanismes d’optimisations utilisés

- Nous allons ensuite rentrer dans le vif du sujet : Le Q-learning, appliqué au jeu vidéo Pong. Nous allons pour cela réaliser une interface permettant à notre algorithme d’intéragir avec le jeu, pour lui permettre d’apprendre. Afin de nous faciliter la tâche, nous utiliserons l’outil TensorFlow (bibliothèque Python), qui permet de faire des calculs de machine learning de façon optimisée
