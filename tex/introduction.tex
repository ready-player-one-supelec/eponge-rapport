\section*{Introduction}
\addcontentsline{toc}{section}{Introduction}
\fancyhead[R]{\textit{Introduction}}

Le projet long \textsc{Ready Player One} a pour but d’étudier le fonctionnement d’algorithmes d'apprentissage automatique. Ce projet d'un an, orientée recherche et développement, cherche à appliquer une branche du machine learning, le Q-Learning, à l’intelligence artificielle (IA)
du jeu vidéo \textsc{Pong}. Cette IA se formera par elle-même sur ce jeu.

Le projet est mené par deux groupes de quatre étudiants, afin de pouvoir comparer les performances des deux produits finaux. 
Notre groupe, nommé « Éponge », est composé de Nathan \textsc{Cassereau}, Raphaël \textsc{Bolut}, Thomas \textsc{Estiez}, et 
Paul \textsc{Le Grand Des Cloizeaux}. 

Les deux groupes sont encadrés par Joanna \textsc{Tomasik} et Arpad \textsc{Rimmel}, qui nous guident et nous donnent des pistes
pour assurer l'avancée du projet, et à qui nous rendont compte chaque semaine du travail réalisé.

L’étude du projet se fait en plusieurs parties. Comme la tâche à réaliser est importante, et que le projet a pour but 
de nous apprendre les mécanismes du machine learning, nous étudierons plusieurs algorithmes différents au cours de l’année, dont nous expérimenterons
le fonctionnement. 
Les différents codes utilisés lors de ce projet sont consultables sur \href{https://github.com/ready-player-one-supelec}{GitHub}.

Dans un premier temps, nous allons étudier le fonctionnement du perceptron, un réseau de neurones basique, que nous allons entraîner 
à la reconnaissance de chiffres manuscrits de la base de données MNIST de Yann \textsc{LeCun}. Cette première étude a pour but de nous faire 
comprendre le fonctionnement global du machine learning, et les différents mécanismes d’optimisations utilisés. Nous vous présenterons les résultats du perceptron que nous avons codé en Python.
 
Puis nous étudierons les réseaux neuronaux à convolution, version améliorée du perceptron. Ces réseaux sont particulièrement adaptés à l'analyse
de certaines données comme les images en couleur.

Nous allons ensuite rentrer dans le vif du sujet : le Q-learning puis le Deep Q-Learning, appliqué au jeu vidéo \textsc{Pong}. 
Nous allons pour cela nous interfacer avec une bibliothèque Python grâce à laquelle notre algorithme pourra agir sur le jeu. Afin de nous faciliter la tâche, nous 
utiliserons l’outil TensorFlow (bibliothèque Python), qui permet de faire des calculs de machine learning de façon optimisée et de nous affranchir
des difficultés d'implémentation posées par le réseau à convolution et sa rétropropagation.
